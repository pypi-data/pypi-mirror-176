\documentclass[a4paper,oneside]{book}
\newenvironment{abstract}{}{}
\usepackage{abstract}
\usepackage{noweb}
% Needed to relax penalty for breaking code chunks across pages, otherwise 
% there might be a lot of space following a code chunk.
\def\nwendcode{\endtrivlist \endgroup}
\let\nwdocspar=\smallbreak

\usepackage[hyphens]{url}
\usepackage{hyperref}
\usepackage{authblk}

\usepackage{mathrsfs}
\usepackage{pdflscape}
\usepackage{changepage}
\newenvironment{examplenotebook}{\adjustwidth{-2em}{}}{\endadjustwidth}
\definecolor{nbsphinxin}{HTML}{FFFFFF}
\definecolor{nbsphinxout}{HTML}{FFFFFF}
\addto\captionsenglish{\renewcommand{\contentsname}{Table of Contents}}


\title{%
  A LADOK3 Python API
}
\author{%
  Alexander Baltatzis,
  Daniel Bosk,
  Gerald Q.\ \enquote{Chip} Maguire Jr.
}
\affil{%
  KTH EECS\\
  \texttt{\{alba,dbosk,maguire\}@kth.se}
}

\begin{document}
\frontmatter
\maketitle

\vspace*{\fill}
\VerbatimInput{../LICENSE}
\clearpage

\begin{abstract}
  \JINJA{from 'inc/person_list.tex' import render_person_lists}


\JINJA{macro render_abstract(abstract, track_class, contrib_type, management=false, status=none, track_judgments=none)}
    %% header

    \setlength{\headheight}{20pt}
    \pagestyle{fancy}
    \renewcommand{\headrulewidth}{0pt}


    {
        \small
        \sffamily
        \noindent
        \VAR{_('Abstract ID')} : \textbf {\VAR{abstract.friendly_id}}
    }

    \vspace{2em}

    \begin{center}
        \textbf {
            \LARGE
            \sffamily
            \VAR{abstract.title}
        }
    \end{center}

    \vspace{1em}

    {\bf
        \noindent
        \large
        \VAR{_('Content')}
    }

    \begin{addmargin}[1em]{1em}
        %% Markdown content
        \rmfamily {
            \allsectionsfont{\rmfamily}
            \sectionfont{\normalsize\rmfamily}
            \subsectionfont{\small\rmfamily}
            \small
            \VAR{abstract.description | markdown}
        }
        \vspace{1.5em}
    \end{addmargin}

    \JINJA{for field in abstract.event.contribution_fields if field.is_active and (management or field.is_public)}
        {\bf
            \noindent
            \large
            \VAR{field.title}
        }

        \begin{addmargin}[1em]{1em}
            %% Markdown content
            \rmfamily {
                \allsectionsfont{\rmfamily}
                \sectionfont{\normalsize\rmfamily}
                \subsectionfont{\small\rmfamily}
                \small
                \VAR{
                    abstract.data_by_field[field.id].friendly_data|markdown
                    if field.id in abstract.data_by_field
                    else ''
                }
            }
            \vspace{1.5em}
        \end{addmargin}
    \JINJA{endfor}

    \vspace{1.5em}

    \VAR{render_person_lists(abstract) | rawlatex}

    \JINJA{if track_class}
    {
        {
            \bf
            \noindent \VAR{_('Track Classification')}:
        }
        \VAR{track_class}
    }
    \JINJA{endif}

    \JINJA{if contrib_type}
    {
        {
            \bf
            \noindent \VAR{_('Contribution Type')}:
        }
        \VAR{contrib_type.name}
    }
    \JINJA{endif}

    \JINJA{if abstract.submission_comment}
        \textbf{\VAR{_('Comments:')}}
        \begin{addmargin}[1em]{1em}
            \VAR{abstract.submission_comment}
        \end{addmargin}
    \JINJA{endif}

    \JINJA{if management}
        \vspace{1em}

        \textbf{\VAR{_('Status:')}} \VAR{status}
        \vspace{2em}

        \setdescription{leftmargin=2em,labelindent=2em}

        \JINJA{if track_judgments}
            \textbf{\VAR{_('Track Reviews:')}}
            \begin{addmargin}[1em]{1em}
                \JINJA{for track_name, judgment, details in track_judgments if judgment}
                    \small \textbf{\VAR{track_name}:}
                    \begin{addmargin}[1em]{1em}
                        \vspace{0.5em}

                        \textbf{\VAR{_('Judgments:')}} \VAR{judgment}

                        \JINJA{if details}
                            \vspace{0.5em}
                            \textbf{\VAR{_("Reviews:")}}
                            \JINJA{for proposed_action, reviewer, comment, score, ratings in details}
                                \vspace{0.5em}
                                \begin{addmargin}[1em]{1em}
                                    \rmfamily {
                                        \allsectionsfont{\rmfamily}
                                        \sectionfont{\normalsize\rmfamily}
                                        \subsectionfont{\small\rmfamily}
                                        \small
                                        \VAR{reviewer}:
                                        \textbf{ \VAR{proposed_action}
                                            \JINJA{if score}
                                                \textbullet\ \VAR{score}
                                            \JINJA{endif}
                                        }
                                        \JINJA{if comment}
                                            (\VAR{comment|markdown})
                                        \JINJA{endif}
                                        \JINJA{if ratings}
                                            \JINJA{for question, rating in ratings}
                                                \vspace{0.5em}
                                                \begin{addmargin}[1em]{1em}
                                                \textit{\VAR{question}} \VAR{rating}
                                                \end{addmargin}
                                            \JINJA{endfor}
                                        \JINJA{endif}
                                    }
                                \end{addmargin}
                            \JINJA{endfor}
                        \JINJA{endif}
                    \end{addmargin}
                \JINJA{endfor}
            \end{addmargin}
        \JINJA{endif}
    \JINJA{endif}

    \VAR{(_('Submitted by {0} on {1}')|latex(true)).format(
        '\\textbf{%s}'|rawlatex|format(abstract.submitter.get_full_name(abbrev_first_name=false, show_title=true)),
        '\\textbf{%s}'|rawlatex|format(abstract.submitted_dt|format_date('full'))
    )|rawlatex}
    \JINJA{if abstract.modified_dt}
        \fancyfoot[C]{\color{gray} \VAR{_('Last modified:')} \VAR{abstract.modified_dt|format_date('full')}}
    \JINJA{endif}
\JINJA{endmacro}

\end{abstract}
\clearpage

\tableofcontents
\clearpage

\mainmatter
\chapter{Introduction}

LADOK (abbreviation of \foreignlanguage{swedish}{Lokalt Adb–baserat 
DOKumentationssystem}, in Swedish) is the national documentation system for 
higher education in Sweden.
This is the documented source code of \texttt{ladok3}, a LADOK3 API wrapper for 
Python.

The \texttt{ladok3} library provides a non-GUI application that, similar to 
access via a web browser, only provides the user with access to the LADOK data 
and functionality that they would actually have based on their specific user 
permissions in LADOK.
It can be seem as a very streamlined web browser just for LADOK's web 
interface.
While the library exploits caching to reduce the load on the LADOK server, this 
represents a subset of the information that would otherwise be obtained via
LADOK's web GUI export functions.

You can install the \texttt{ladok3} package by running
\begin{minted}{bash}
pip3 install ladok3
\end{minted}
in the terminal.
You can find a quick reference by running
\begin{minted}[firstnumber=last]{bash}
pydoc ladok3
\end{minted}

We provide the main class \texttt{LadokSession} (\cref{LadokSession}).
The \texttt{LadokSession} class acts like \enquote{factories} and will return 
objects representing various LADOK data.
These data objects' classes inherit the \texttt{LadokData} (\cref{LadokData}) 
and \texttt{LadokRemoteData} (\cref{LadokRemoteData}) classes.
Data of the type \texttt{LadokData} is not expected to change, unlike 
\texttt{LadokRemoteData}, which is.
Objects of type \texttt{LadokRemoteData} know how to update themselves, \ie fetch 
and refresh their data from LADOK.
When relevant they can also write data to LADOK, \ie update entries such as 
results.

One design criteria is to improve performance.
We do this by caching all factory methods of any \texttt{LadokSession}.
The \texttt{LadokSession} itself is also designed to be cacheable: if the session to 
LADOK expires, it will automatically reauthenticate to establish a new session.



\part{The library}

\documentclass[a4paper,oneside]{book}
\newenvironment{abstract}{}{}
\usepackage{abstract}
\usepackage{noweb}
% Needed to relax penalty for breaking code chunks across pages, otherwise 
% there might be a lot of space following a code chunk.
\def\nwendcode{\endtrivlist \endgroup}
\let\nwdocspar=\smallbreak

\usepackage[hyphens]{url}
\usepackage{hyperref}
\usepackage{authblk}

\usepackage{mathrsfs}
\usepackage{pdflscape}
\usepackage{changepage}
\newenvironment{examplenotebook}{\adjustwidth{-2em}{}}{\endadjustwidth}
\definecolor{nbsphinxin}{HTML}{FFFFFF}
\definecolor{nbsphinxout}{HTML}{FFFFFF}
\addto\captionsenglish{\renewcommand{\contentsname}{Table of Contents}}


\title{%
  A LADOK3 Python API
}
\author{%
  Alexander Baltatzis,
  Daniel Bosk,
  Gerald Q.\ \enquote{Chip} Maguire Jr.
}
\affil{%
  KTH EECS\\
  \texttt{\{alba,dbosk,maguire\}@kth.se}
}

\begin{document}
\frontmatter
\maketitle

\vspace*{\fill}
\VerbatimInput{../LICENSE}
\clearpage

\begin{abstract}
  \JINJA{from 'inc/person_list.tex' import render_person_lists}


\JINJA{macro render_abstract(abstract, track_class, contrib_type, management=false, status=none, track_judgments=none)}
    %% header

    \setlength{\headheight}{20pt}
    \pagestyle{fancy}
    \renewcommand{\headrulewidth}{0pt}


    {
        \small
        \sffamily
        \noindent
        \VAR{_('Abstract ID')} : \textbf {\VAR{abstract.friendly_id}}
    }

    \vspace{2em}

    \begin{center}
        \textbf {
            \LARGE
            \sffamily
            \VAR{abstract.title}
        }
    \end{center}

    \vspace{1em}

    {\bf
        \noindent
        \large
        \VAR{_('Content')}
    }

    \begin{addmargin}[1em]{1em}
        %% Markdown content
        \rmfamily {
            \allsectionsfont{\rmfamily}
            \sectionfont{\normalsize\rmfamily}
            \subsectionfont{\small\rmfamily}
            \small
            \VAR{abstract.description | markdown}
        }
        \vspace{1.5em}
    \end{addmargin}

    \JINJA{for field in abstract.event.contribution_fields if field.is_active and (management or field.is_public)}
        {\bf
            \noindent
            \large
            \VAR{field.title}
        }

        \begin{addmargin}[1em]{1em}
            %% Markdown content
            \rmfamily {
                \allsectionsfont{\rmfamily}
                \sectionfont{\normalsize\rmfamily}
                \subsectionfont{\small\rmfamily}
                \small
                \VAR{
                    abstract.data_by_field[field.id].friendly_data|markdown
                    if field.id in abstract.data_by_field
                    else ''
                }
            }
            \vspace{1.5em}
        \end{addmargin}
    \JINJA{endfor}

    \vspace{1.5em}

    \VAR{render_person_lists(abstract) | rawlatex}

    \JINJA{if track_class}
    {
        {
            \bf
            \noindent \VAR{_('Track Classification')}:
        }
        \VAR{track_class}
    }
    \JINJA{endif}

    \JINJA{if contrib_type}
    {
        {
            \bf
            \noindent \VAR{_('Contribution Type')}:
        }
        \VAR{contrib_type.name}
    }
    \JINJA{endif}

    \JINJA{if abstract.submission_comment}
        \textbf{\VAR{_('Comments:')}}
        \begin{addmargin}[1em]{1em}
            \VAR{abstract.submission_comment}
        \end{addmargin}
    \JINJA{endif}

    \JINJA{if management}
        \vspace{1em}

        \textbf{\VAR{_('Status:')}} \VAR{status}
        \vspace{2em}

        \setdescription{leftmargin=2em,labelindent=2em}

        \JINJA{if track_judgments}
            \textbf{\VAR{_('Track Reviews:')}}
            \begin{addmargin}[1em]{1em}
                \JINJA{for track_name, judgment, details in track_judgments if judgment}
                    \small \textbf{\VAR{track_name}:}
                    \begin{addmargin}[1em]{1em}
                        \vspace{0.5em}

                        \textbf{\VAR{_('Judgments:')}} \VAR{judgment}

                        \JINJA{if details}
                            \vspace{0.5em}
                            \textbf{\VAR{_("Reviews:")}}
                            \JINJA{for proposed_action, reviewer, comment, score, ratings in details}
                                \vspace{0.5em}
                                \begin{addmargin}[1em]{1em}
                                    \rmfamily {
                                        \allsectionsfont{\rmfamily}
                                        \sectionfont{\normalsize\rmfamily}
                                        \subsectionfont{\small\rmfamily}
                                        \small
                                        \VAR{reviewer}:
                                        \textbf{ \VAR{proposed_action}
                                            \JINJA{if score}
                                                \textbullet\ \VAR{score}
                                            \JINJA{endif}
                                        }
                                        \JINJA{if comment}
                                            (\VAR{comment|markdown})
                                        \JINJA{endif}
                                        \JINJA{if ratings}
                                            \JINJA{for question, rating in ratings}
                                                \vspace{0.5em}
                                                \begin{addmargin}[1em]{1em}
                                                \textit{\VAR{question}} \VAR{rating}
                                                \end{addmargin}
                                            \JINJA{endfor}
                                        \JINJA{endif}
                                    }
                                \end{addmargin}
                            \JINJA{endfor}
                        \JINJA{endif}
                    \end{addmargin}
                \JINJA{endfor}
            \end{addmargin}
        \JINJA{endif}
    \JINJA{endif}

    \VAR{(_('Submitted by {0} on {1}')|latex(true)).format(
        '\\textbf{%s}'|rawlatex|format(abstract.submitter.get_full_name(abbrev_first_name=false, show_title=true)),
        '\\textbf{%s}'|rawlatex|format(abstract.submitted_dt|format_date('full'))
    )|rawlatex}
    \JINJA{if abstract.modified_dt}
        \fancyfoot[C]{\color{gray} \VAR{_('Last modified:')} \VAR{abstract.modified_dt|format_date('full')}}
    \JINJA{endif}
\JINJA{endmacro}

\end{abstract}
\clearpage

\tableofcontents
\clearpage

\mainmatter
\chapter{Introduction}

LADOK (abbreviation of \foreignlanguage{swedish}{Lokalt Adb–baserat 
DOKumentationssystem}, in Swedish) is the national documentation system for 
higher education in Sweden.
This is the documented source code of \texttt{ladok3}, a LADOK3 API wrapper for 
Python.

The \texttt{ladok3} library provides a non-GUI application that, similar to 
access via a web browser, only provides the user with access to the LADOK data 
and functionality that they would actually have based on their specific user 
permissions in LADOK.
It can be seem as a very streamlined web browser just for LADOK's web 
interface.
While the library exploits caching to reduce the load on the LADOK server, this 
represents a subset of the information that would otherwise be obtained via
LADOK's web GUI export functions.

You can install the \texttt{ladok3} package by running
\begin{minted}{bash}
pip3 install ladok3
\end{minted}
in the terminal.
You can find a quick reference by running
\begin{minted}[firstnumber=last]{bash}
pydoc ladok3
\end{minted}

We provide the main class \texttt{LadokSession} (\cref{LadokSession}).
The \texttt{LadokSession} class acts like \enquote{factories} and will return 
objects representing various LADOK data.
These data objects' classes inherit the \texttt{LadokData} (\cref{LadokData}) 
and \texttt{LadokRemoteData} (\cref{LadokRemoteData}) classes.
Data of the type \texttt{LadokData} is not expected to change, unlike 
\texttt{LadokRemoteData}, which is.
Objects of type \texttt{LadokRemoteData} know how to update themselves, \ie fetch 
and refresh their data from LADOK.
When relevant they can also write data to LADOK, \ie update entries such as 
results.

One design criteria is to improve performance.
We do this by caching all factory methods of any \texttt{LadokSession}.
The \texttt{LadokSession} itself is also designed to be cacheable: if the session to 
LADOK expires, it will automatically reauthenticate to establish a new session.



\part{The library}

\documentclass[a4paper,oneside]{book}
\newenvironment{abstract}{}{}
\usepackage{abstract}
\usepackage{noweb}
% Needed to relax penalty for breaking code chunks across pages, otherwise 
% there might be a lot of space following a code chunk.
\def\nwendcode{\endtrivlist \endgroup}
\let\nwdocspar=\smallbreak

\usepackage[hyphens]{url}
\usepackage{hyperref}
\usepackage{authblk}

\usepackage{mathrsfs}
\usepackage{pdflscape}
\usepackage{changepage}
\newenvironment{examplenotebook}{\adjustwidth{-2em}{}}{\endadjustwidth}
\definecolor{nbsphinxin}{HTML}{FFFFFF}
\definecolor{nbsphinxout}{HTML}{FFFFFF}
\addto\captionsenglish{\renewcommand{\contentsname}{Table of Contents}}


\title{%
  A LADOK3 Python API
}
\author{%
  Alexander Baltatzis,
  Daniel Bosk,
  Gerald Q.\ \enquote{Chip} Maguire Jr.
}
\affil{%
  KTH EECS\\
  \texttt{\{alba,dbosk,maguire\}@kth.se}
}

\begin{document}
\frontmatter
\maketitle

\vspace*{\fill}
\VerbatimInput{../LICENSE}
\clearpage

\begin{abstract}
  \JINJA{from 'inc/person_list.tex' import render_person_lists}


\JINJA{macro render_abstract(abstract, track_class, contrib_type, management=false, status=none, track_judgments=none)}
    %% header

    \setlength{\headheight}{20pt}
    \pagestyle{fancy}
    \renewcommand{\headrulewidth}{0pt}


    {
        \small
        \sffamily
        \noindent
        \VAR{_('Abstract ID')} : \textbf {\VAR{abstract.friendly_id}}
    }

    \vspace{2em}

    \begin{center}
        \textbf {
            \LARGE
            \sffamily
            \VAR{abstract.title}
        }
    \end{center}

    \vspace{1em}

    {\bf
        \noindent
        \large
        \VAR{_('Content')}
    }

    \begin{addmargin}[1em]{1em}
        %% Markdown content
        \rmfamily {
            \allsectionsfont{\rmfamily}
            \sectionfont{\normalsize\rmfamily}
            \subsectionfont{\small\rmfamily}
            \small
            \VAR{abstract.description | markdown}
        }
        \vspace{1.5em}
    \end{addmargin}

    \JINJA{for field in abstract.event.contribution_fields if field.is_active and (management or field.is_public)}
        {\bf
            \noindent
            \large
            \VAR{field.title}
        }

        \begin{addmargin}[1em]{1em}
            %% Markdown content
            \rmfamily {
                \allsectionsfont{\rmfamily}
                \sectionfont{\normalsize\rmfamily}
                \subsectionfont{\small\rmfamily}
                \small
                \VAR{
                    abstract.data_by_field[field.id].friendly_data|markdown
                    if field.id in abstract.data_by_field
                    else ''
                }
            }
            \vspace{1.5em}
        \end{addmargin}
    \JINJA{endfor}

    \vspace{1.5em}

    \VAR{render_person_lists(abstract) | rawlatex}

    \JINJA{if track_class}
    {
        {
            \bf
            \noindent \VAR{_('Track Classification')}:
        }
        \VAR{track_class}
    }
    \JINJA{endif}

    \JINJA{if contrib_type}
    {
        {
            \bf
            \noindent \VAR{_('Contribution Type')}:
        }
        \VAR{contrib_type.name}
    }
    \JINJA{endif}

    \JINJA{if abstract.submission_comment}
        \textbf{\VAR{_('Comments:')}}
        \begin{addmargin}[1em]{1em}
            \VAR{abstract.submission_comment}
        \end{addmargin}
    \JINJA{endif}

    \JINJA{if management}
        \vspace{1em}

        \textbf{\VAR{_('Status:')}} \VAR{status}
        \vspace{2em}

        \setdescription{leftmargin=2em,labelindent=2em}

        \JINJA{if track_judgments}
            \textbf{\VAR{_('Track Reviews:')}}
            \begin{addmargin}[1em]{1em}
                \JINJA{for track_name, judgment, details in track_judgments if judgment}
                    \small \textbf{\VAR{track_name}:}
                    \begin{addmargin}[1em]{1em}
                        \vspace{0.5em}

                        \textbf{\VAR{_('Judgments:')}} \VAR{judgment}

                        \JINJA{if details}
                            \vspace{0.5em}
                            \textbf{\VAR{_("Reviews:")}}
                            \JINJA{for proposed_action, reviewer, comment, score, ratings in details}
                                \vspace{0.5em}
                                \begin{addmargin}[1em]{1em}
                                    \rmfamily {
                                        \allsectionsfont{\rmfamily}
                                        \sectionfont{\normalsize\rmfamily}
                                        \subsectionfont{\small\rmfamily}
                                        \small
                                        \VAR{reviewer}:
                                        \textbf{ \VAR{proposed_action}
                                            \JINJA{if score}
                                                \textbullet\ \VAR{score}
                                            \JINJA{endif}
                                        }
                                        \JINJA{if comment}
                                            (\VAR{comment|markdown})
                                        \JINJA{endif}
                                        \JINJA{if ratings}
                                            \JINJA{for question, rating in ratings}
                                                \vspace{0.5em}
                                                \begin{addmargin}[1em]{1em}
                                                \textit{\VAR{question}} \VAR{rating}
                                                \end{addmargin}
                                            \JINJA{endfor}
                                        \JINJA{endif}
                                    }
                                \end{addmargin}
                            \JINJA{endfor}
                        \JINJA{endif}
                    \end{addmargin}
                \JINJA{endfor}
            \end{addmargin}
        \JINJA{endif}
    \JINJA{endif}

    \VAR{(_('Submitted by {0} on {1}')|latex(true)).format(
        '\\textbf{%s}'|rawlatex|format(abstract.submitter.get_full_name(abbrev_first_name=false, show_title=true)),
        '\\textbf{%s}'|rawlatex|format(abstract.submitted_dt|format_date('full'))
    )|rawlatex}
    \JINJA{if abstract.modified_dt}
        \fancyfoot[C]{\color{gray} \VAR{_('Last modified:')} \VAR{abstract.modified_dt|format_date('full')}}
    \JINJA{endif}
\JINJA{endmacro}

\end{abstract}
\clearpage

\tableofcontents
\clearpage

\mainmatter
\chapter{Introduction}

LADOK (abbreviation of \foreignlanguage{swedish}{Lokalt Adb–baserat 
DOKumentationssystem}, in Swedish) is the national documentation system for 
higher education in Sweden.
This is the documented source code of \texttt{ladok3}, a LADOK3 API wrapper for 
Python.

The \texttt{ladok3} library provides a non-GUI application that, similar to 
access via a web browser, only provides the user with access to the LADOK data 
and functionality that they would actually have based on their specific user 
permissions in LADOK.
It can be seem as a very streamlined web browser just for LADOK's web 
interface.
While the library exploits caching to reduce the load on the LADOK server, this 
represents a subset of the information that would otherwise be obtained via
LADOK's web GUI export functions.

You can install the \texttt{ladok3} package by running
\begin{minted}{bash}
pip3 install ladok3
\end{minted}
in the terminal.
You can find a quick reference by running
\begin{minted}[firstnumber=last]{bash}
pydoc ladok3
\end{minted}

We provide the main class \texttt{LadokSession} (\cref{LadokSession}).
The \texttt{LadokSession} class acts like \enquote{factories} and will return 
objects representing various LADOK data.
These data objects' classes inherit the \texttt{LadokData} (\cref{LadokData}) 
and \texttt{LadokRemoteData} (\cref{LadokRemoteData}) classes.
Data of the type \texttt{LadokData} is not expected to change, unlike 
\texttt{LadokRemoteData}, which is.
Objects of type \texttt{LadokRemoteData} know how to update themselves, \ie fetch 
and refresh their data from LADOK.
When relevant they can also write data to LADOK, \ie update entries such as 
results.

One design criteria is to improve performance.
We do this by caching all factory methods of any \texttt{LadokSession}.
The \texttt{LadokSession} itself is also designed to be cacheable: if the session to 
LADOK expires, it will automatically reauthenticate to establish a new session.



\part{The library}

\documentclass[a4paper,oneside]{book}
\newenvironment{abstract}{}{}
\usepackage{abstract}
\usepackage{noweb}
% Needed to relax penalty for breaking code chunks across pages, otherwise 
% there might be a lot of space following a code chunk.
\def\nwendcode{\endtrivlist \endgroup}
\let\nwdocspar=\smallbreak

\usepackage[hyphens]{url}
\usepackage{hyperref}
\usepackage{authblk}

\input{preamble.tex}

\title{%
  A LADOK3 Python API
}
\author{%
  Alexander Baltatzis,
  Daniel Bosk,
  Gerald Q.\ \enquote{Chip} Maguire Jr.
}
\affil{%
  KTH EECS\\
  \texttt{\{alba,dbosk,maguire\}@kth.se}
}

\begin{document}
\frontmatter
\maketitle

\vspace*{\fill}
\VerbatimInput{../LICENSE}
\clearpage

\begin{abstract}
  \input{abstract.tex}
\end{abstract}
\clearpage

\tableofcontents
\clearpage

\mainmatter
\chapter{Introduction}

LADOK (abbreviation of \foreignlanguage{swedish}{Lokalt Adb–baserat 
DOKumentationssystem}, in Swedish) is the national documentation system for 
higher education in Sweden.
This is the documented source code of \texttt{ladok3}, a LADOK3 API wrapper for 
Python.

The \texttt{ladok3} library provides a non-GUI application that, similar to 
access via a web browser, only provides the user with access to the LADOK data 
and functionality that they would actually have based on their specific user 
permissions in LADOK.
It can be seem as a very streamlined web browser just for LADOK's web 
interface.
While the library exploits caching to reduce the load on the LADOK server, this 
represents a subset of the information that would otherwise be obtained via
LADOK's web GUI export functions.

You can install the \texttt{ladok3} package by running
\begin{minted}{bash}
pip3 install ladok3
\end{minted}
in the terminal.
You can find a quick reference by running
\begin{minted}[firstnumber=last]{bash}
pydoc ladok3
\end{minted}

We provide the main class \texttt{LadokSession} (\cref{LadokSession}).
The \texttt{LadokSession} class acts like \enquote{factories} and will return 
objects representing various LADOK data.
These data objects' classes inherit the \texttt{LadokData} (\cref{LadokData}) 
and \texttt{LadokRemoteData} (\cref{LadokRemoteData}) classes.
Data of the type \texttt{LadokData} is not expected to change, unlike 
\texttt{LadokRemoteData}, which is.
Objects of type \texttt{LadokRemoteData} know how to update themselves, \ie fetch 
and refresh their data from LADOK.
When relevant they can also write data to LADOK, \ie update entries such as 
results.

One design criteria is to improve performance.
We do this by caching all factory methods of any \texttt{LadokSession}.
The \texttt{LadokSession} itself is also designed to be cacheable: if the session to 
LADOK expires, it will automatically reauthenticate to establish a new session.



\part{The library}

\input{../src/ladok3/ladok3.tex}


\part{Logging in at different universities}

\input{../src/ladok3/kth.tex}


\part{API calls}

\input{../src/ladok3/api.tex}
\input{../src/ladok3/undoc.tex}



\part{A command-line interface}

\chapter{The base interface}

\input{../src/ladok3/cli.tex}

\chapter{The \texttt{data} command}

\input{../src/ladok3/data.tex}

\chapter{The \texttt{report} command}

\input{../src/ladok3/report.tex}

\chapter{The \texttt{student} command}

\input{../src/ladok3/student.tex}



\part{Other example applications}

\chapter{Transfer results from KTH Canvas to LADOK}

Here we provide an example program~\texttt{canvas2ladok.py} which exports 
results from KTH Canvas to LADOK for the introductory programming course~prgi 
(DD1315).
You can find an up-to-date version of this chapter at
\begin{center}
  \url{https://github.com/dbosk/intropy/tree/master/adm/reporting}.
\end{center}

\input{../examples/canvas2ladok.tex}


\backmatter
\printbibliography


\end{document}



\part{Logging in at different universities}

\input{../src/ladok3/kth.tex}


\part{API calls}

\input{../src/ladok3/api.tex}
\input{../src/ladok3/undoc.tex}



\part{A command-line interface}

\chapter{The base interface}

\input{../src/ladok3/cli.tex}

\chapter{The \texttt{data} command}

TEST FILE


\chapter{The \texttt{report} command}

\JINJA{extends 'inc/document.tex'}
\JINJA{from 'inc/first_page.tex' import render_first_page}
\JINJA{from 'inc/contribution.tex' import render_contribution}
\JINJA{from 'inc/abstract.tex' import render_abstract}


\JINJA{block document_class}
    \documentclass[a4paper, 11pt, oneside]{book}
\JINJA{endblock}


\JINJA{block geometry}
    \usepackage[a4paper, top=5em, bottom=10em]{geometry}
\JINJA{endblock}


\JINJA{block header_extra}
    \usepackage{fancyhdr}
    \usepackage{scrextend}

    \setlength{\headheight}{60pt}
    \pagestyle{fancy}
    \renewcommand{\headrulewidth}{0pt}
\JINJA{endblock}

\JINJA{block content}
    %% first page
    \frontmatter
    \VAR{render_first_page(event, title, logo_img=logo_img, url=url)|rawlatex}

    %% body
    \mainmatter

    \JINJA{for item in items}
        \newpage
        \fancyhead[L]{\small \rmfamily \color{gray} \truncateellipses{\VAR{event.title}}{80pt} / \VAR{title}}
        \fancyhead[R]{\small \rmfamily \color{gray} \truncateellipses{\VAR{item.title}}{150pt}}
        \phantomsection
        \addcontentsline{toc}{section}{\VAR{item.title}}

        \JINJA{if doc_type == 'abstract'}
            \JINJA{if management}
                \VAR{render_abstract(item, get_track_classification(item), get_contrib_type(item), management=true,
                                     status=get_status(item), track_judgments=get_track_judgments(item)) | rawlatex}
            \JINJA{else}
                \VAR{render_abstract(item, get_track_classification(item), get_contrib_type(item)) | rawlatex}
            \JINJA{endif}
        \JINJA{elif doc_type == 'contribution'}
            \VAR{render_contribution(item, tz) | rawlatex}
        \JINJA{endif}

        \fancyfoot[L]{\small \rmfamily \color{gray} \today}
        \fancyfoot[C]{}
        \fancyfoot[R]{\small \rmfamily \color{gray} \VAR{(_('Page {}')|latex(true)).format('\\thepage')|rawlatex }}
    \JINJA{endfor}
\JINJA{endblock}


\chapter{The \texttt{student} command}

\input{../src/ladok3/student.tex}



\part{Other example applications}

\chapter{Transfer results from KTH Canvas to LADOK}

Here we provide an example program~\texttt{canvas2ladok.py} which exports 
results from KTH Canvas to LADOK for the introductory programming course~prgi 
(DD1315).
You can find an up-to-date version of this chapter at
\begin{center}
  \url{https://github.com/dbosk/intropy/tree/master/adm/reporting}.
\end{center}

\input{../examples/canvas2ladok.tex}


\backmatter
\printbibliography


\end{document}



\part{Logging in at different universities}

\input{../src/ladok3/kth.tex}


\part{API calls}

\input{../src/ladok3/api.tex}
\input{../src/ladok3/undoc.tex}



\part{A command-line interface}

\chapter{The base interface}

\input{../src/ladok3/cli.tex}

\chapter{The \texttt{data} command}

TEST FILE


\chapter{The \texttt{report} command}

\JINJA{extends 'inc/document.tex'}
\JINJA{from 'inc/first_page.tex' import render_first_page}
\JINJA{from 'inc/contribution.tex' import render_contribution}
\JINJA{from 'inc/abstract.tex' import render_abstract}


\JINJA{block document_class}
    \documentclass[a4paper, 11pt, oneside]{book}
\JINJA{endblock}


\JINJA{block geometry}
    \usepackage[a4paper, top=5em, bottom=10em]{geometry}
\JINJA{endblock}


\JINJA{block header_extra}
    \usepackage{fancyhdr}
    \usepackage{scrextend}

    \setlength{\headheight}{60pt}
    \pagestyle{fancy}
    \renewcommand{\headrulewidth}{0pt}
\JINJA{endblock}

\JINJA{block content}
    %% first page
    \frontmatter
    \VAR{render_first_page(event, title, logo_img=logo_img, url=url)|rawlatex}

    %% body
    \mainmatter

    \JINJA{for item in items}
        \newpage
        \fancyhead[L]{\small \rmfamily \color{gray} \truncateellipses{\VAR{event.title}}{80pt} / \VAR{title}}
        \fancyhead[R]{\small \rmfamily \color{gray} \truncateellipses{\VAR{item.title}}{150pt}}
        \phantomsection
        \addcontentsline{toc}{section}{\VAR{item.title}}

        \JINJA{if doc_type == 'abstract'}
            \JINJA{if management}
                \VAR{render_abstract(item, get_track_classification(item), get_contrib_type(item), management=true,
                                     status=get_status(item), track_judgments=get_track_judgments(item)) | rawlatex}
            \JINJA{else}
                \VAR{render_abstract(item, get_track_classification(item), get_contrib_type(item)) | rawlatex}
            \JINJA{endif}
        \JINJA{elif doc_type == 'contribution'}
            \VAR{render_contribution(item, tz) | rawlatex}
        \JINJA{endif}

        \fancyfoot[L]{\small \rmfamily \color{gray} \today}
        \fancyfoot[C]{}
        \fancyfoot[R]{\small \rmfamily \color{gray} \VAR{(_('Page {}')|latex(true)).format('\\thepage')|rawlatex }}
    \JINJA{endfor}
\JINJA{endblock}


\chapter{The \texttt{student} command}

\input{../src/ladok3/student.tex}



\part{Other example applications}

\chapter{Transfer results from KTH Canvas to LADOK}

Here we provide an example program~\texttt{canvas2ladok.py} which exports 
results from KTH Canvas to LADOK for the introductory programming course~prgi 
(DD1315).
You can find an up-to-date version of this chapter at
\begin{center}
  \url{https://github.com/dbosk/intropy/tree/master/adm/reporting}.
\end{center}

\input{../examples/canvas2ladok.tex}


\backmatter
\printbibliography


\end{document}



\part{Logging in at different universities}

\input{../src/ladok3/kth.tex}


\part{API calls}

\input{../src/ladok3/api.tex}
\input{../src/ladok3/undoc.tex}



\part{A command-line interface}

\chapter{The base interface}

\input{../src/ladok3/cli.tex}

\chapter{The \texttt{data} command}

TEST FILE


\chapter{The \texttt{report} command}

\JINJA{extends 'inc/document.tex'}
\JINJA{from 'inc/first_page.tex' import render_first_page}
\JINJA{from 'inc/contribution.tex' import render_contribution}
\JINJA{from 'inc/abstract.tex' import render_abstract}


\JINJA{block document_class}
    \documentclass[a4paper, 11pt, oneside]{book}
\JINJA{endblock}


\JINJA{block geometry}
    \usepackage[a4paper, top=5em, bottom=10em]{geometry}
\JINJA{endblock}


\JINJA{block header_extra}
    \usepackage{fancyhdr}
    \usepackage{scrextend}

    \setlength{\headheight}{60pt}
    \pagestyle{fancy}
    \renewcommand{\headrulewidth}{0pt}
\JINJA{endblock}

\JINJA{block content}
    %% first page
    \frontmatter
    \VAR{render_first_page(event, title, logo_img=logo_img, url=url)|rawlatex}

    %% body
    \mainmatter

    \JINJA{for item in items}
        \newpage
        \fancyhead[L]{\small \rmfamily \color{gray} \truncateellipses{\VAR{event.title}}{80pt} / \VAR{title}}
        \fancyhead[R]{\small \rmfamily \color{gray} \truncateellipses{\VAR{item.title}}{150pt}}
        \phantomsection
        \addcontentsline{toc}{section}{\VAR{item.title}}

        \JINJA{if doc_type == 'abstract'}
            \JINJA{if management}
                \VAR{render_abstract(item, get_track_classification(item), get_contrib_type(item), management=true,
                                     status=get_status(item), track_judgments=get_track_judgments(item)) | rawlatex}
            \JINJA{else}
                \VAR{render_abstract(item, get_track_classification(item), get_contrib_type(item)) | rawlatex}
            \JINJA{endif}
        \JINJA{elif doc_type == 'contribution'}
            \VAR{render_contribution(item, tz) | rawlatex}
        \JINJA{endif}

        \fancyfoot[L]{\small \rmfamily \color{gray} \today}
        \fancyfoot[C]{}
        \fancyfoot[R]{\small \rmfamily \color{gray} \VAR{(_('Page {}')|latex(true)).format('\\thepage')|rawlatex }}
    \JINJA{endfor}
\JINJA{endblock}


\chapter{The \texttt{student} command}

\input{../src/ladok3/student.tex}



\part{Other example applications}

\chapter{Transfer results from KTH Canvas to LADOK}

Here we provide an example program~\texttt{canvas2ladok.py} which exports 
results from KTH Canvas to LADOK for the introductory programming course~prgi 
(DD1315).
You can find an up-to-date version of this chapter at
\begin{center}
  \url{https://github.com/dbosk/intropy/tree/master/adm/reporting}.
\end{center}

\input{../examples/canvas2ladok.tex}


\backmatter
\printbibliography


\end{document}
